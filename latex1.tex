\documentclass{article}
\usepackage{graphicx}
\usepackage{lipsum}

\title{Tytuł Twojej Pracy}
\author{Imię Nazwisko}
\date{\today}

\begin{document}

\maketitle

\begin{abstract}
    To jest streszczenie pracy.
\end{abstract}

\section{Wprowadzenie}
Tekst wprowadzenia.

\section{Rozdział Pierwszy}
\subsection{Podrozdział 1.1}
Tekst podrozdziału.

\subsection{Podrozdział 1.2}
Tekst podrozdziału.

\section{Rozdział Drugi}
\subsection{Podrozdział 2.1}
Tekst podrozdziału.

\subsection{Podrozdział 2.2}
Tekst podrozdziału.

\section{Formatowanie Tekstu}
\textbf{Tekst pogrubiony}, \emph{tekst pochylony}, \underline{tekst podkreślony}.

\section{Tryb Matematyczny}
Przykład równania w trybie matematycznym: $E=mc^2$.

\section{Rysunki}
\begin{figure}[h]
    \centering
    \includegraphics[width=0.5\textwidth]{example-image-a}
    \caption{Podpis do pierwszego rysunku.}
\end{figure}

\begin{figure}[h]
    \centering
    \begin{tabular}{c}
        \includegraphics[width=0.3\textwidth]{example-image-b} \\
        \textbf{Rysunek z tekstem.}
    \end{tabular}
    \caption{Podpis do drugiego rysunku.}
\end{figure}

\section{Tabela}
\begin{table}[h]
    \centering
    \begin{tabular}{|c|c|}
        \hline
        Kolumna 1 & Kolumna 2 \\
        \hline
        Wiersz 1  & Wartość A \\
        \hline
        Wiersz 2  & Wartość B \\
        \hline
    \end{tabular}
    \caption{Tabela odnosząca się do tekstu.}
\end{table}

\section{Odwołania do Rysunków i Tabel}
Jak widzimy na Rysunku \ref{fig:rysunek1}, oraz na Rysunku \ref{fig:rysunek2} z tekstem, tabela \ref{tab:tabela1} przedstawia...

\section{Podsumowanie}
Podsumowanie pracy.

\begin{thebibliography}{9}
    \bibitem{przyklad1} Autor A. \emph{Tytuł Publikacji A.} Wydawnictwo, 2022.
    \bibitem{przyklad2} Autor B. \emph{Tytuł Publikacji B.} Inne Wydawnictwo, 2023.
\end{thebibliography}

\end{document}