\documentclass{beamer}
\usepackage{graphicx}
\usetheme{Madrid}
\usecolortheme{seahorse}

\title{Tytuł Prezentacji}
\author{Imię Nazwisko}
\date{\today}

\begin{document}

\begin{frame}
  \titlepage
\end{frame}

\section{Wprowadzenie}
\begin{frame}{Wprowadzenie}
  Tekst wprowadzenia.
\end{frame}

\section{Motywacja}
\begin{frame}{Motywacja}
  \begin{itemize}
    \item Element dynamiczny 1.
    \item Element dynamiczny 2.
  \end{itemize}
\end{frame}

\section{Rysunki}
\begin{frame}{Rysunek 1}
  \begin{figure}
    \centering
    \includegraphics[width=0.7\textwidth]{example-image-a}
    \caption{Podpis do rysunku 1.}
  \end{figure}
\end{frame}

\begin{frame}{Rysunek 2}
  \begin{figure}
    \centering
    \includegraphics[width=0.7\textwidth]{example-image-b}
    \caption{Podpis do rysunku 2.}
  \end{figure}
\end{frame}

\section{Tabela}
\begin{frame}{Tabela}
  \begin{table}
    \centering
    \begin{tabular}{|c|c|}
      \hline
      Kolumna 1 & Kolumna 2 \\
      \hline
      Wiersz 1   & Wartość A \\
      Wiersz 2   & Wartość B \\
      \hline
    \end{tabular}
    \caption{Tabela odnosząca się do prezentacji.}
  \end{table}
\end{frame}

\section{Elementy Dynamiczne}
\begin{frame}{Elementy Dynamiczne}
  \begin{enumerate}
    \item Pierwszy element dynamiczny.
    \item Drugi element dynamiczny.
  \end{enumerate}
\end{frame}

\section{Odwołania}
\begin{frame}{Odwołania}
  Odwołanie do Rysunku \ref{fig:rysunek1} oraz Rysunku \ref{fig:rysunek2}. Tabela \ref{tab:tabela1} przedstawia...
\end{frame}

\section{Bibliografia}
\begin{frame}{Bibliografia}
  \begin{thebibliography}{9}
    \bibitem{przyklad1} Autor A. \emph{Tytuł Publikacji A.} Wydawnictwo, 2022.
    \bibitem{przyklad2} Autor B. \emph{Tytuł Publikacji B.} Inne Wydawnictwo, 2023.
  \end{thebibliography}
\end{frame}

\end{document}